\documentclass[]{scrartcl}

%opening
\title{Shape From Shading}
\author{Marta Pibiri, 65175\\Francesca Cella, 65172}

\begin{document}

\maketitle

\section{Introduzione}
Tramite tecniche di computer vision è possibile ottenere la superficie 3D di un oggetto avendo a disposizione delle rappresentazioni 2D dello stesso, ciascuna con le rispettive informazioni sull'illuminazione. La tecnica scelta per questo esperimento è la \textit{Photometric Stereo}, la quale consiste nell'utilizzo di una sola telecamera, con posizione fissa, e diverse fonti di illuminazione.\\%da riformulare, è inteso come come diverse angolazioni di illuminazione
L'esperimento effettuato è diviso in due parti: nella prima parte, sono stati implementati due metodi di fattorizzazione QR, quello di Givens e quello di Householder; nella seconda parte, è stata effettuata la risoluzione del sistema che permette di stabilire le normali della superficie a partire dalle immagini e dalle luci in input. %integrare

\section{Legge di Lambert}
La legge di Lambert mette in relazione la riflettanza di una superficie, ovvero il suo albedo (che dipende dal materiale), con le sue normali e le informazioni sulla luce. La legge si esprime come:\\%riformulare?
\begin{center}
	$i$ = $\alpha$ $\langle$ \underline{$n$}, \underline{$l$} $\rangle$
\end{center}
%altre informazioni
Se un oggetto viene illuminato di fronte, si vede bene, ma se viene illuminato di lato è buoio: al crescere dell'angolo formatosi tra la posizione dell'osservatore e la fonte luminosa, l'intensità della riflessione diminuisce. %aggiungere magari un disegno e spiegare meglio
Si suppone che la luce abbia distanza infinta dall'oggetto.

\section{Formulazione del problema}
Il problema può essere formulato come sistema utilizzando le seguenti matrici:
\begin{center}
	$DN\textsuperscript{T}L = M$
\end{center}
dove D è la matrice diagonale contenente i valori dell'albedo, N è la matrice contenente le normali, nonché incognita e obiettivo dell'esperimento, L è la matrice contenente le luci e M è la matrice contenente le foto.\\\\
Nel nostro esperimento, per semplicità, supponiamo che la matrice degli albedo sia una matrice identità di dimensione $p \times p$.\\
I vettori che rappresentano le normali possono essere rappresentate tramite una matrice contenente $p$ righe, ovvero una riga per ogni pixel.\\ %da rivedere
I vettori che rappresentano le luci possono essere espressi come matrice $3 \times q$.\\
L'insieme delle immagini 2D che rappresentano l'oggetto può essere considerato come una matrice di pixel $p \times q$.\\\\
Tramite la legge di Lambert è possibile esprimere una misura per ogni pixel: possiamo quindi riformulare il problema considerando come matrice delle misure, $M$, una matrice contenente per ogni colonna una delle $k$ foto considerate.

\section{Prima parte} %trovare un altro titolo
\section{Seconda parte}%anche per questo
\end{document}
